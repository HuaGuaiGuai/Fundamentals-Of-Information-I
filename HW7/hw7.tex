%
% This is a borrowed LaTeX template file for lecture notes for CS267,
% Applications of Parallel Computing, UCBerkeley EECS Department.
% Now being used for CMU's 10725 Fall 2012 Optimization course
% taught by Geoff Gordon and Ryan Tibshirani.  When preparing
% LaTeX notes for this class, please use this template.
%
% To familiarize yourself with this template, the body contains
% some examples of its use.  Look them over.  Then you can
% run LaTeX on this file.  After you have LaTeXed this file then
% you can look over the result either by printing it out with
% dvips or using xdvi. "pdflatex template.tex" should also work.
%

\documentclass[UTF8,oneside]{article}

% \usepackage[UTF8,scheme=plain]{ctex}
\usepackage[AutoFakeBold,AutoFakeSlant,CJKecglue]{xeCJK}  % 载入 xeCJK以支持中文,支持伪粗体,伪斜体 , 去掉CJK 文字与西文字体间的空格
\usepackage[margin=1in]{geometry}
\usepackage{amsmath,amsthm,amssymb}
\usepackage{graphicx}
\usepackage{autobreak}
\usepackage{tikz}
\usepackage{array}
\usetikzlibrary{positioning} %为了实现相对位置的设定
\usepackage{xcolor} %为了实现不同的颜色
\setCJKmainfont{宋体}                                         % 设置中文中文字体
\setCJKmonofont{宋体}                                        % 设置中文等宽字体
% \setCJKsansfont{宋体}
% \setCJKmainfont{SimSun}[BoldFont=SimHei, ItalicFont=KaiTi]
\setlength{\oddsidemargin}{0.25 in}
\setlength{\evensidemargin}{-0.25 in}
\setlength{\topmargin}{-0.6 in}
\setlength{\textwidth}{6.5 in}
\setlength{\textheight}{8.5 in}
\setlength{\headsep}{0.75 in}
\setlength{\parindent}{0 in}
\setlength{\parskip}{0.1 in}

%
% ADD PACKAGES here:
%

\usepackage{amsmath,amsfonts,graphicx}

%
% The following commands set up the lecnum (lecture number)
% counter and make various numbering schemes work relative
% to the lecture number.
%
\newcounter{lecnum}
\renewcommand{\thepage}{\thelecnum-\arabic{page}}
\renewcommand{\thesection}{\thelecnum.\arabic{section}}
\renewcommand{\theequation}{\thelecnum.\arabic{equation}}
\renewcommand{\thefigure}{\thelecnum.\arabic{figure}}
\renewcommand{\thetable}{\thelecnum.\arabic{table}}

%
% The following macro is used to generate the header.
%
\newcommand{\lecture}[4]{
   \pagestyle{myheadings}
   \thispagestyle{plain}
   \newpage
   \setcounter{lecnum}{#1}
   \setcounter{page}{1}
   \noindent
   \begin{center}
   \framebox{
      \vbox{\vspace{2mm}
    \hbox to 6.28in { {\bf Fundamentals Of Information Science
	\hfill 2022 Spring} }
       \vspace{4mm}
       \hbox to 6.28in { {\Large \hfill   #2  \hfill} }
       \vspace{2mm}
       \hbox to 6.28in { {\it 学生: #3 \hfill 时间: #4} }
      \vspace{2mm}}
   }
   \end{center}
   \markboth{Lecture #1: #2}{Lecture #1: #2}

}
%
% Convention for citations is authors' initials followed by the year.
% For example, to cite a paper by Leighton and Maggs you would type
% \cite{LM89}, and to cite a paper by Strassen you would type \cite{S69}.
% (To avoid bibliography problems, for now we redefine the \cite command.)
% Also commands that create a suitable format for the reference list.
\renewcommand{\cite}[1]{[#1]}
\def\beginrefs{\begin{list}%
        {[\arabic{equation}]}{\usecounter{equation}
         \setlength{\leftmargin}{2.0truecm}\setlength{\labelsep}{0.4truecm}%
         \setlength{\labelwidth}{1.6truecm}}}
\def\endrefs{\end{list}}
\def\bibentry#1{\item[\hbox{[#1]}]}

%Use this command for a figure; it puts a figure in wherever you want it.
%usage: \fig{NUMBER}{SPACE-IN-INCHES}{CAPTION}
\newcommand{\fig}[3]{
			\vspace{#2}
			\begin{center}
			Figure \thelecnum.#1:~#3
			\end{center}
	}
% Use these for theorems, lemmas, proofs, etc.
\usepackage{amsthm}
\newtheorem*{Solution}{Solution}
\newtheorem{theorem}{Theorem}[lecnum]
\newtheorem{lemma}[theorem]{Lemma}

\newtheorem{proposition}[theorem]{Proposition}
\newtheorem{claim}[theorem]{Claim}
\newtheorem{corollary}[theorem]{Corollary}
\newtheorem{definition}[theorem]{Definition}
% \newenvironment{proof}{{\bf Proof:}}{\hfill\rule{2mm}{2mm}}

% **** IF YOU WANT TO DEFINE ADDITIONAL MACROS FOR YOURSELF, PUT THEM HERE:

\newcommand\E{\mathbb{E}}

\begin{document}
%FILL IN THE RIGHT INFO.
%\lecture{**LECTURE-NUMBER**}{**DATE**}{**LECTURER**}{**SCRIBE**}
\lecture{1}{Homework7}{华园(202000120027))}{2022.3.23}
%\footnotetext{These notes are partially based on those of Nigel Mansell.}

% **** YOUR NOTES GO HERE:

% Some general latex examples and examples making use of the
% macros follow.
%**** IN GENERAL, BE BRIEF. LONG SCRIBE NOTES, NO MATTER HOW WELL WRITTEN,
%**** ARE NEVER READ BY ANYBODY.

\section*{Problem 1.} % Don't be this informal in your notes!
\begin{Solution}
首先介绍我的编程思路之后再出示代码实现(内含注释):\\
\qquad(1)parity-check矩阵的生成,我利用Python的一个包pyldpc中的生成H的方法,进行了直接的生成。\\
\qquad(2)00.....000000BSC信道之后接收到的信息我们用一个数组R表示,如何生成R,我写了一个按照概率生成随机数的函数进行实现,从而我们获得了接受到的信息,每次接收到的具有随机性。\\
\qquad(3)对于先验概率L(V),我通过遍历R中元素,根据元素是1还是0进行判断,并修改其数值为$log(\frac{1-p}{p})$和$log(\frac{p}{1-p})$,从而获得了先验概率的一维数组。\\\qquad
\qquad(4)解码部分可分解为两部分:variable-nodes向check-nodes传递信息和check-nodes向variable-nodes传递信息。针对这部分内容,我构造了M矩阵和E矩阵,其中M矩阵代表variable-nodes传递的信息,而E矩阵代表check-nodes传递的信息,这两个矩阵均与H矩阵同型,迭代的过程就是对这两个矩阵进行更新的过程。对M进行更新的过程,要对E矩阵进行列求和,并且减去对应位置的数值。而对E矩阵更新的过程就是对M矩阵进行一定的行操作。\\
\qquad(5)每进行一次迭代,根据后验概率进行判断,获得解码的码字,如果不满足$H·C^T=0$,则进行迭代,如果满足,直接跳出,若在规定的最大迭代次数内没有正确解码,则视为无法解码。\\
\qquad(6)输入不同的p值进行测试,绘图,基本思路可以由下图表示:
\begin{center}
\includegraphics[scale=0.20]{variable.png}
\includegraphics[scale=0.20]{check.png}
\end{center}
整体编程思路可用下图总结:(摘选自B站网课)
\begin{center}
\includegraphics[scale=0.30]{A.png}
\end{center}
下面给出代码:
(1)生成随机数方法以及整体代码:(目录中附了代码,学长学姐看不清可以直接看代码)
\begin{center}
\includegraphics[scale=0.40]{random.png}
\end{center}
\begin{center}
\includegraphics[scale=0.20,width=400pt]{code.png}
\end{center}
(2)图像:
\begin{center}
\includegraphics[scale=0.350]{result.png}
\end{center}
(3)最大迭代数设置为:20;生成校验矩阵和生成矩阵的时间约为0.5s,针对每个p进行1000次解码,并且最大迭代数字为20,由于python解1000个码字还是比较吃力地,因此相应耗时比较长,cpu时间大约分为两类,(1)当可解码的时候(p<=0.04),每个p时长大约为38分钟.(2)完全不可解码后(p>=0.05),每个p时长平均4.16个小时左右
	
\end{Solution}
\end{document}


