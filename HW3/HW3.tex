%
% This is a borrowed LaTeX template file for lecture notes for CS267,
% Applications of Parallel Computing, UCBerkeley EECS Department.
% Now being used for CMU's 10725 Fall 2012 Optimization course
% taught by Geoff Gordon and Ryan Tibshirani.  When preparing
% LaTeX notes for this class, please use this template.
%
% To familiarize yourself with this template, the body contains
% some examples of its use.  Look them over.  Then you can
% run LaTeX on this file.  After you have LaTeXed this file then
% you can look over the result either by printing it out with
% dvips or using xdvi. "pdflatex template.tex" should also work.
%

\documentclass[UTF8,oneside]{article}

% \usepackage[UTF8,scheme=plain]{ctex}
\usepackage[AutoFakeBold,AutoFakeSlant,CJKecglue]{xeCJK}  % 载入 xeCJK以支持中文,支持伪粗体,伪斜体 , 去掉CJK 文字与西文字体间的空格
\usepackage[margin=1in]{geometry}
\usepackage{amsmath,amsthm,amssymb}
\usepackage{graphicx}
\usepackage{autobreak}
\usepackage{tikz}
\usetikzlibrary{positioning} %为了实现相对位置的设定
\usepackage{xcolor} %为了实现不同的颜色
\setCJKmainfont{宋体}                                         % 设置中文中文字体
\setCJKmonofont{宋体}                                        % 设置中文等宽字体
% \setCJKsansfont{宋体}
% \setCJKmainfont{SimSun}[BoldFont=SimHei, ItalicFont=KaiTi]
\setlength{\oddsidemargin}{0.25 in}
\setlength{\evensidemargin}{-0.25 in}
\setlength{\topmargin}{-0.6 in}
\setlength{\textwidth}{6.5 in}
\setlength{\textheight}{8.5 in}
\setlength{\headsep}{0.75 in}
\setlength{\parindent}{0 in}
\setlength{\parskip}{0.1 in}

%
% ADD PACKAGES here:
%

\usepackage{amsmath,amsfonts,graphicx}

%
% The following commands set up the lecnum (lecture number)
% counter and make various numbering schemes work relative
% to the lecture number.
%
\newcounter{lecnum}
\renewcommand{\thepage}{\thelecnum-\arabic{page}}
\renewcommand{\thesection}{\thelecnum.\arabic{section}}
\renewcommand{\theequation}{\thelecnum.\arabic{equation}}
\renewcommand{\thefigure}{\thelecnum.\arabic{figure}}
\renewcommand{\thetable}{\thelecnum.\arabic{table}}

%
% The following macro is used to generate the header.
%
\newcommand{\lecture}[4]{
   \pagestyle{myheadings}
   \thispagestyle{plain}
   \newpage
   \setcounter{lecnum}{#1}
   \setcounter{page}{1}
   \noindent
   \begin{center}
   \framebox{
      \vbox{\vspace{2mm}
    \hbox to 6.28in { {\bf Fundamentals Of Information Science
	\hfill 2022 Spring} }
       \vspace{4mm}
       \hbox to 6.28in { {\Large \hfill Homework3  \hfill} }
       \vspace{2mm}
       \hbox to 6.28in { {\it 学生: #3 \hfill 时间: #4} }
      \vspace{2mm}}
   }
   \end{center}
   \markboth{Lecture #1: #2}{Lecture #1: #2}

}
%
% Convention for citations is authors' initials followed by the year.
% For example, to cite a paper by Leighton and Maggs you would type
% \cite{LM89}, and to cite a paper by Strassen you would type \cite{S69}.
% (To avoid bibliography problems, for now we redefine the \cite command.)
% Also commands that create a suitable format for the reference list.
\renewcommand{\cite}[1]{[#1]}
\def\beginrefs{\begin{list}%
        {[\arabic{equation}]}{\usecounter{equation}
         \setlength{\leftmargin}{2.0truecm}\setlength{\labelsep}{0.4truecm}%
         \setlength{\labelwidth}{1.6truecm}}}
\def\endrefs{\end{list}}
\def\bibentry#1{\item[\hbox{[#1]}]}

%Use this command for a figure; it puts a figure in wherever you want it.
%usage: \fig{NUMBER}{SPACE-IN-INCHES}{CAPTION}
\newcommand{\fig}[3]{
			\vspace{#2}
			\begin{center}
			Figure \thelecnum.#1:~#3
			\end{center}
	}
% Use these for theorems, lemmas, proofs, etc.
\usepackage{amsthm}
\newtheorem*{Solution}{Solution}
\newtheorem{theorem}{Theorem}[lecnum]
\newtheorem{lemma}[theorem]{Lemma}

\newtheorem{proposition}[theorem]{Proposition}
\newtheorem{claim}[theorem]{Claim}
\newtheorem{corollary}[theorem]{Corollary}
\newtheorem{definition}[theorem]{Definition}
% \newenvironment{proof}{{\bf Proof:}}{\hfill\rule{2mm}{2mm}}

% **** IF YOU WANT TO DEFINE ADDITIONAL MACROS FOR YOURSELF, PUT THEM HERE:

\newcommand\E{\mathbb{E}}

\begin{document}
%FILL IN THE RIGHT INFO.
%\lecture{**LECTURE-NUMBER**}{**DATE**}{**LECTURER**}{**SCRIBE**}
\lecture{1}{信息与比特}{华园(202000120027))}{2022.3.9}
%\footnotetext{These notes are partially based on those of Nigel Mansell.}

% **** YOUR NOTES GO HERE:

% Some general latex examples and examples making use of the
% macros follow.
%**** IN GENERAL, BE BRIEF. LONG SCRIBE NOTES, NO MATTER HOW WELL WRITTEN,
%**** ARE NEVER READ BY ANYBODY.

\section*{Problem 1.} % Don't be this informal in your notes!
Coin flips.A fair coin is flipped until the first head occurs. Let X denote the number of flips required. Find the 
entropy H(X) in bits.\\
\begin{Solution}
Suppose the probability distribution of random variable X is P(X).
\begin{align*}
P(X=1) &=\frac{1}{2} \\
P(X=2) &=(\frac{1}{2})^2   \\
P(X=3) &=(\frac{1}{2})^3   \\
P(X=4) &=(\frac{1}{2})^4   \\
\cdots       \\
P(X=n) &=(\frac{1}{2})^n   \\
\end{align*}
According to the definition of entropy, we can get the following results:
\end{Solution}
\begin{align*}
H(X) &=-\sum_{n=1}^\infty p(x)log_{2}p(x) \\
&=-\sum_{n=1}^\infty(\frac{1}{2})^nlog_{2}(\frac{1}{2})^n\\
&=\sum_{n=1}^\infty n(\frac{1}{2})^n\\
&=\frac{1/2}{(1-1/2)^2}\\
&=2(bits)
\end{align*}

\section*{Problem 2.} % Don't be this informal in your notes!
A die comes up 6 twice as often as it comes up 1.What is the maximum entropy?
\begin{Solution}
Suppose the probability distribution of random variable X is P(X).
\begin{align*}
P(X=1) &=x \\
P(X=2) &=\frac{1-3x}{4}   \\
P(X=3) &=\frac{1-3x}{4}  \\
P(X=4) &=\frac{1-3x}{4}   \\
P(X=5) &=\frac{1-3x}{4}  \\
P(X=6) &=2x  \\
\end{align*}
According to the definition of entropy, we can get the following results:
\end{Solution}
\begin{align*}
H(X) &=-\sum_{n=1}^\infty p(x)lnp(x) \\
&=-(xlnx+2xlnx+(1-3x)ln(\frac{1-3x}{4}))
\end{align*}
thus
\begin{align*}
\frac{\mathrm{d}H(X)}{\mathrm{d}X} &=-(xlnx+2xlnx+(1-3x)ln(\frac{1-3x}{4}))\\
&=-(lnx+1+2ln2x+2-3ln(\frac{1-3x}{4})-3)\\
&=-(lnx+2lnx+2ln2-3ln(1-3x)+3ln4)\\
&=-(3ln(\frac{x}{1-3x})+4ln4)\\
\\
Let \frac{\mathrm{d}H(X)}{\mathrm{d}X}=0:\qquad\qquad&we\; can\; know\;that:\quad x_0=\frac{1}{3+2^{\frac{8}{3}}}\\
\\
&when\; x<x_0,\quad\frac{\mathrm{d}H(X)}{\mathrm{d}X}>0\\
&when\;x>x_0,\quad\frac{\mathrm{d}H(X)}{\mathrm{d}X}<0\\
\\
thus\;the&\; maximum\;entropy\;is\;H(x_0)
\end{align*}

\begin{align*}
H(X)_{max}=H(x_0)=\frac{1}{3+2^{\frac{8}{3}}}ln(3+2^{\frac{8}{3}})+\frac{2}{3+2^{\frac{8}{3}}}ln\frac{3+2^{\frac{8}{3}}}{2}-\frac{2^{\frac{8}{3}}}{3+2^{\frac{8}{3}}}ln\frac{2^{\frac{2}{3}}}{3+2^{\frac{8}{3}}}
\end{align*}


\section*{Problem 3.}
AEP and source coding. A discrete memoryless source emits a sequence of statistically
independent binary digits with probabilities p(1) = 0.005 and p(0) = 0.995. The digits are
taken 100 at a time and a binary codeword is provided for every sequence of 100 digits
containing three or fewer 1’s.\\
\\
(a) Assuming that all codewords are the same length, find the minimum length required to provide codewords for all sequences with three or fewer 1’s.
\begin{Solution}
Assuming that the number of all sequences with three or fewer 1’s is N.
\begin{align*}
N&=C_{100}^{1}+ C_{100}^{2}+C_{100}^{3}+C_{100}^{0}\\
&=100+\frac{(100·99)}{2}+\frac{(100·99·98)}{3·2}+1\\
&=166751\\
&={(10\quad1000\quad1011\quad0101\quad1111)}_{B}
\end{align*}
\qquad\qquad In summary,the minimum length is 18.
\end{Solution}
(b) Calculate the probability of observing a source sequence for which no codeword has been assigned.
\begin{Solution}
Assuming that the probability of observing a source sequence for which no codeword has
been assigned is P.
\begin{align*}
P&=1-C_{100}^{0}*(0.995)^{100}-C_{100}^{1}*(0.995)^{99}*0.005-C_{100}^{2}*(0.995)^{98}*(0.005)^{2}-C_{100}^{3}*(0.995)^{97}*(0.005)^{3}\\
&=1-0.60577-0.304407-0.075719-0.012429\\
&\approx 0.0017
\end{align*}
\end{Solution}
(c) Use Chebyshev’s inequality (search online if you don’t know Chebyshev’s inequality)
to bound the probability of observing a source sequence for which no codeword has been
assigned. Compare this bound with the actual probability computed in part (b).
\begin{Solution}
Suppose that the random variable X represents the number of 1 in the sequence
\begin{align*}
X\sim {B(100,0.005)},E(X)=0.5,D(X)&=0.4975\\
P(|X-E(X)|\geq3.5)&\leq\frac{D(X)}{(3.5)^2}\\
P(|X-E(X)|\geq3.5)&\leq0.04061\\
\\
0.0017<0.04061&
\end{align*}
From this, we can get that the estimation of Chebyshev inequality is relatively rough, and there is a certain gap with the actual probability value.
\end{Solution}

\section*{Problem 4.}
Entropy and pairwise independence. Let X, Y, Z be three binary Bernoulli random variables that are pairwise independent; that is, I(X; Y ) = I(X;Z) = I(Y ;Z) = 0.\\
(a) Under this constraint, what is the minimum value for H(X, Y, Z)?
\begin{Solution}
\end{Solution}
$$ X=Y=Z=\left\{
\begin{aligned}
 & & 1 \quad P= \frac{1}{2}\\
 & & 0 \quad P= \frac{1}{2}\\
\end{aligned}
\right.
$$
\begin{center}
\begin{align*}
H(X)=H(Y)=H(Z)&=\frac{1}{2}log_2(2)+\frac{1}{2}log_2(2)=1\\
\\
H(X,Y,Z)&=H(X)+H(Y|X)+H(Z|Y,X)\\
&=H(X)+H(Y)+H(Z|Y,X)\\
\\
when\;H(Z|Y,X)=0,\;\;&H(X)\; is\; the\; minimum.\\
H(X)_{min}&=2

\end{align*}
\end{center}
(b) Give an example achieving this minimum.
\begin{Solution}
\end{Solution}
Suppose a letter is randomly selected from A, B, C and D, and the probability of each letter is 0.25 ,Now set up the following events:\\
\begin{center}
Event X: The extracted letter is A or B, \quad P(X)=0.5.\\
Event Y: The extracted letter is A or C, \quad P(Y)=0.5.\\
Event Z: The extracted letter is B or C, \quad P(Z)=0.5.\\
\end{center}
We assume that random variables X, Y and Z correspond to event X, event Y and event Z respectively. X = 1 means that event X occurs, X= 0 means that event X does not occur, and so on.
\begin{center}
(1).X,Y,Z are three binary Bernoulli(0.5) random variables.\\
(2).P(X,Y)=P(X)P(Y),P(X,Z)=P(X)P(Z),P(Z,Y)=P(Z)P(Y)\\
so X,Y,Z are pairwise independent.
\begin{align*}
when \;(X=0,Y=0):\;P(Z=1)=0,\;P(Z=0)=1.&\qquad  [P(X=0,Y=0)=\frac{1}{4}]\\
when \;(X=0,Y=1):\;P(Z=1)=1,\;P(Z=0)=0. &\qquad [P(X=0,Y=1)=\frac{1}{4}]\\
when \;(X=1,Y=0):\;P(Z=1)=1,\;P(Z=0)=0. &\qquad [P(X=1,Y=0)=\frac{1}{4}]\\
when \;(X=1,Y=1):\;P(Z=1)=0,\;P(Z=0)=1. &\qquad [P(X=1,Y=1)=\frac{1}{4}]\\
\\
H(Z|Y,X)=\frac{1}{4}log_2(1)+\frac{1}{4}log_2(1)+\frac{1}{4}log_2(1)+&\frac{1}{4}log_2(1)=0\\
\\
H(X,Y,Z)=H(X)+H(Y)=2.(achieving\; &the\; minimum)
\end{align*}
\end{center}
% **** THIS ENDS THE EXAMPLES. DON'T DELETE THE FOLLOWING LINE:

\end{document}





