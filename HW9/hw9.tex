%
% This is a borrowed LaTeX template file for lecture notes for CS267,
% Applications of Parallel Computing, UCBerkeley EECS Department.
% Now being used for CMU's 10725 Fall 2012 Optimization course
% taught by Geoff Gordon and Ryan Tibshirani.  When preparing
% LaTeX notes for this class, please use this template.
%
% To familiarize yourself with this template, the body contains
% some examples of its use.  Look them over.  Then you can
% run LaTeX on this file.  After you have LaTeXed this file then
% you can look over the result either by printing it out with
% dvips or using xdvi. "pdflatex template.tex" should also work.
%

\documentclass[UTF8,oneside]{article}

% \usepackage[UTF8,scheme=plain]{ctex}
\usepackage[AutoFakeBold,AutoFakeSlant,CJKecglue]{xeCJK}  % 载入 xeCJK以支持中文,支持伪粗体,伪斜体 , 去掉CJK 文字与西文字体间的空格
\usepackage[margin=1in]{geometry}
\usepackage{amsmath,amsthm,amssymb}
\usepackage{graphicx}
\usepackage{autobreak}
\usepackage{tikz}
\usepackage{array}
\usetikzlibrary{positioning} %为了实现相对位置的设定
\usepackage{xcolor} %为了实现不同的颜色
\setCJKmainfont{宋体}                                         % 设置中文中文字体
\setCJKmonofont{宋体}                                        % 设置中文等宽字体
% \setCJKsansfont{宋体}
% \setCJKmainfont{SimSun}[BoldFont=SimHei, ItalicFont=KaiTi]
\setlength{\oddsidemargin}{0.25 in}
\setlength{\evensidemargin}{-0.25 in}
\setlength{\topmargin}{-0.6 in}
\setlength{\textwidth}{6.5 in}
\setlength{\textheight}{8.5 in}
\setlength{\headsep}{0.75 in}
\setlength{\parindent}{0 in}
\setlength{\parskip}{0.1 in}

%
% ADD PACKAGES here:
%

\usepackage{amsmath,amsfonts,graphicx}

%
% The following commands set up the lecnum (lecture number)
% counter and make various numbering schemes work relative
% to the lecture number.
%
\newcounter{lecnum}
\renewcommand{\thepage}{\thelecnum-\arabic{page}}
\renewcommand{\thesection}{\thelecnum.\arabic{section}}
\renewcommand{\theequation}{\thelecnum.\arabic{equation}}
\renewcommand{\thefigure}{\thelecnum.\arabic{figure}}
\renewcommand{\thetable}{\thelecnum.\arabic{table}}

%
% The following macro is used to generate the header.
%
\newcommand{\lecture}[4]{
   \pagestyle{myheadings}
   \thispagestyle{plain}
   \newpage
   \setcounter{lecnum}{#1}
   \setcounter{page}{1}
   \noindent
   \begin{center}
   \framebox{
      \vbox{\vspace{2mm}
    \hbox to 6.28in { {\bf Fundamentals Of Information Science
	\hfill 2022 Spring} }
       \vspace{4mm}
       \hbox to 6.28in { {\Large \hfill   #2  \hfill} }
       \vspace{2mm}
       \hbox to 6.28in { {\it 学生: #3 \hfill 时间: #4} }
      \vspace{2mm}}
   }
   \end{center}
   \markboth{Lecture #1: #2}{Lecture #1: #2}

}
%
% Convention for citations is authors' initials followed by the year.
% For example, to cite a paper by Leighton and Maggs you would type
% \cite{LM89}, and to cite a paper by Strassen you would type \cite{S69}.
% (To avoid bibliography problems, for now we redefine the \cite command.)
% Also commands that create a suitable format for the reference list.
\renewcommand{\cite}[1]{[#1]}
\def\beginrefs{\begin{list}%
        {[\arabic{equation}]}{\usecounter{equation}
         \setlength{\leftmargin}{2.0truecm}\setlength{\labelsep}{0.4truecm}%
         \setlength{\labelwidth}{1.6truecm}}}
\def\endrefs{\end{list}}
\def\bibentry#1{\item[\hbox{[#1]}]}

%Use this command for a figure; it puts a figure in wherever you want it.
%usage: \fig{NUMBER}{SPACE-IN-INCHES}{CAPTION}
\newcommand{\fig}[3]{
			\vspace{#2}
			\begin{center}
			Figure \thelecnum.#1:~#3
			\end{center}
	}
% Use these for theorems, lemmas, proofs, etc.
\usepackage{amsthm}
\newtheorem*{Solution}{Solution}
\newtheorem{theorem}{Theorem}[lecnum]
\newtheorem{lemma}[theorem]{Lemma}

\newtheorem{proposition}[theorem]{Proposition}
\newtheorem{claim}[theorem]{Claim}
\newtheorem{corollary}[theorem]{Corollary}
\newtheorem{definition}[theorem]{Definition}
% \newenvironment{proof}{{\bf Proof:}}{\hfill\rule{2mm}{2mm}}

% **** IF YOU WANT TO DEFINE ADDITIONAL MACROS FOR YOURSELF, PUT THEM HERE:

\newcommand\E{\mathbb{E}}

\begin{document}
%FILL IN THE RIGHT INFO.
%\lecture{**LECTURE-NUMBER**}{**DATE**}{**LECTURER**}{**SCRIBE**}
\lecture{1}{Homework9}{华园(202000120027))}{2022.3.23}
%\footnotetext{These notes are partially based on those of Nigel Mansell.}

% **** YOUR NOTES GO HERE:

% Some general latex examples and examples making use of the
% macros follow.
%**** IN GENERAL, BE BRIEF. LONG SCRIBE NOTES, NO MATTER HOW WELL WRITTEN,
%**** ARE NEVER READ BY ANYBODY.

\section*{Problem 1.} % Don't be this informal in your notes!
multiplicative one-time pad\\
We may also define a "multiplication mod p" variation of the one-time pad. This is a cipher $\mathcal{E}=(E, D)$, defined over $(\mathcal{K}, \mathcal{M}, \mathcal{C})$, where $\mathcal{K}:=\mathcal{M}:=\mathcal{C}:=\{1, \ldots, p-1\}$, where $p$ is a prime. Encryption and decryption are defined as follows:
$$
E(k, m):=k \cdot m \bmod p \quad D(k, c):=k^{-1} \cdot c \bmod p .
$$
Here, $k^{-1}$ denotes the multiplicative inverse of $k$ modulo $p$. Verify the correctness/ property for this cipher and prove that it is perfectly secure.
\begin{Solution}
\end{Solution}
(1)为了验证其正确性,我们仅需要验证D(k,E(k,m))=m,因此我们可以获得
\begin{align*}
E(k,m)&=k·m \;mod \;p\\
D(k,E(k,m))&=k^{-1}·(k·m \;mod\; p)mod \;p\\
&=m 
\end{align*}
因此我们可以验证其正确性。
(2)为了证明是否绝对安全,我们需要验证每个不同的信息被加密成同一密文的概率相同,由于mod乘除运算的特殊性,以及k的等概率分布,我们只需要证明针对不同的m,有且仅有1个k使得m被加密为c.
由于k·m mod p=c,则可以表示为$$k·m=c+np,(n>=0)$$之后的讨论我们只针对一位进行讨论,k<p,则我们一定可以获得的是$$0<=n<=m-1$$,由于k<p,则有$$m·p>c+np$$,c+np mod m=k·m mod m=0,那么我们可以证明对于任意一个信息$m_i$都有一个整数$k=\frac{c+np}{m_i}$,使得$m_i$被加密为c,且其余满足条件的k值为$k=\frac{c+(n+dm_i)p}{m_i}$,均不在范围内,因此可以证明k的唯一性,由于k的等概率性,从而我们可以证明这种加密方式是绝对安全的。\\
经过查阅相关材料,获得乘法逆元又称数论倒数。若  且a,m互质,则x为a的逆元,记为,若a,m不互质,则不存在逆元。当且仅当m为素数时,a有唯一的乘法逆元。在此题目中,P与k互质,且$k^{-1}$为k的逆元,因此我们可以确定,针对任意一个$m_i$均有唯一一个k使得$m_i$被加密为C,因此我们可以获得$$Pr[E(k,m)=c]=constant$$
从而可以证明该加密方法是绝对安全的。
\section*{Problem 2.}
Truncating PRFs\\
Let $F$ be a PRF whose range is $\mathcal{Y}=\{0,1\}^{n}$. For some $\ell<n$ consider the PRF $F^{\prime}$ with a range $\mathcal{Y}^{\prime}=\{0,1\}^{\ell}$ defined as: $F^{\prime}(k, x)=F(k, x)[0 \ldots \ell]$. That is, we truncate the output of $F(k, x)$ to the first $\ell$ bits. Show that if $F$ is a secure PRF then so is $F^{\prime}$.
\begin{Solution}
\end{Solution}
倘若F是安全的PRF,那么可以获得F(k,x)与f(x)无法区分,也就是说从K中随机选取一个k与从所有函数中随机选取一个函数是无法区分的,针对n位无法区分,则针对其前l位,同样无法进行区分,因此F'也是安全的。
\section*{Problem 3.}
Chain encryption\\
Let $\mathcal{E}=(E, D)$ be a perfectly secure cipher defined over $(\mathcal{K}, \mathcal{M}, \mathcal{C})$ where $\mathcal{K}=\mathcal{M}$. Let $\mathcal{E}^{\prime}=$ $\left(E^{\prime}, D^{\prime}\right)$ be a cipher where encryption is defined as $E^{\prime}\left(\left(k_{1}, k_{2}\right), m\right):=\left(E\left(k_{1}, k_{2}\right), E\left(k_{2}, m\right)\right)$. Show that $\mathcal{E}^{\prime}$ is perfectly secure.
\begin{Solution}
\end{Solution}
根据绝对安全的定义,我们可以假设:$$
\forall m_{0}, m_{1} \in M\text { and } \forall c \in C
$$
由于(E,D)是绝对安全的,因此
$$Pr[E(k_2,m_0)=c_1]=Pr[E(k_2,m_1)=c_1]$$
$$Pr[E(k_1,k_2)=c_2]=Pr[E(k_1,k_3)=c_2]$$
也就是说对于$E[k_1,k_2]和E[k_2,m]$两者均为安全加密,从而两者的分布确定,因此可知:
$$Pr[E'((k_1,k_2),m_0)=c]=Pr[E'((k_1,k_2),m_1)=c]$$
从而可以判断$E'((k_1,k_2),m)$是绝对安全的。
% **** THIS ENDS THE EXAMPLES. DON'T DELETE THE FOLLOWING LINE:

\end{document}





