%
% This is a borrowed LaTeX template file for lecture notes for CS267,
% Applications of Parallel Computing, UCBerkeley EECS Department.
% Now being used for CMU's 10725 Fall 2012 Optimization course
% taught by Geoff Gordon and Ryan Tibshirani.  When preparing
% LaTeX notes for this class, please use this template.
%
% To familiarize yourself with this template, the body contains
% some examples of its use.  Look them over.  Then you can
% run LaTeX on this file.  After you have LaTeXed this file then
% you can look over the result either by printing it out with
% dvips or using xdvi. "pdflatex template.tex" should also work.
%

\documentclass[UTF8,oneside]{article}

% \usepackage[UTF8,scheme=plain]{ctex}
\usepackage[AutoFakeBold,AutoFakeSlant,CJKecglue]{xeCJK}  % 载入 xeCJK以支持中文,支持伪粗体,伪斜体 , 去掉CJK 文字与西文字体间的空格
\usepackage[margin=1in]{geometry}
\usepackage{amsmath,amsthm,amssymb}
\usepackage{graphicx}
\usepackage{autobreak}
\usepackage{tikz}
\usetikzlibrary{positioning} %为了实现相对位置的设定
\usepackage{xcolor} %为了实现不同的颜色
\setCJKmainfont{宋体}                                         % 设置中文中文字体
\setCJKmonofont{宋体}                                        % 设置中文等宽字体
% \setCJKsansfont{宋体}
% \setCJKmainfont{SimSun}[BoldFont=SimHei, ItalicFont=KaiTi]
\setlength{\oddsidemargin}{0.25 in}
\setlength{\evensidemargin}{-0.25 in}
\setlength{\topmargin}{-0.6 in}
\setlength{\textwidth}{6.5 in}
\setlength{\textheight}{8.5 in}
\setlength{\headsep}{0.75 in}
\setlength{\parindent}{0 in}
\setlength{\parskip}{0.1 in}

%
% ADD PACKAGES here:
%

\usepackage{amsmath,amsfonts,graphicx}

%
% The following commands set up the lecnum (lecture number)
% counter and make various numbering schemes work relative
% to the lecture number.
%
\newcounter{lecnum}
\renewcommand{\thepage}{\thelecnum-\arabic{page}}
\renewcommand{\thesection}{\thelecnum.\arabic{section}}
\renewcommand{\theequation}{\thelecnum.\arabic{equation}}
\renewcommand{\thefigure}{\thelecnum.\arabic{figure}}
\renewcommand{\thetable}{\thelecnum.\arabic{table}}

%
% The following macro is used to generate the header.
%
\newcommand{\lecture}[4]{
   \pagestyle{myheadings}
   \thispagestyle{plain}
   \newpage
   \setcounter{lecnum}{#1}
   \setcounter{page}{1}
   \noindent
   \begin{center}
   \framebox{
      \vbox{\vspace{2mm}
    \hbox to 6.28in { {\bf Fundamentals Of Information Science
	\hfill 2022 Spring} }
       \vspace{4mm}
       \hbox to 6.28in { {\Large \hfill HW #1: #2  \hfill} }
       \vspace{2mm}
       \hbox to 6.28in { {\it 学生: #3 \hfill 时间: #4} }
      \vspace{2mm}}
   }
   \end{center}
   \markboth{Lecture #1: #2}{Lecture #1: #2}

}
%
% Convention for citations is authors' initials followed by the year.
% For example, to cite a paper by Leighton and Maggs you would type
% \cite{LM89}, and to cite a paper by Strassen you would type \cite{S69}.
% (To avoid bibliography problems, for now we redefine the \cite command.)
% Also commands that create a suitable format for the reference list.
\renewcommand{\cite}[1]{[#1]}
\def\beginrefs{\begin{list}%
        {[\arabic{equation}]}{\usecounter{equation}
         \setlength{\leftmargin}{2.0truecm}\setlength{\labelsep}{0.4truecm}%
         \setlength{\labelwidth}{1.6truecm}}}
\def\endrefs{\end{list}}
\def\bibentry#1{\item[\hbox{[#1]}]}

%Use this command for a figure; it puts a figure in wherever you want it.
%usage: \fig{NUMBER}{SPACE-IN-INCHES}{CAPTION}
\newcommand{\fig}[3]{
			\vspace{#2}
			\begin{center}
			Figure \thelecnum.#1:~#3
			\end{center}
	}
% Use these for theorems, lemmas, proofs, etc.
\usepackage{amsthm}
\newtheorem*{Solution}{Solution}
\newtheorem{theorem}{Theorem}[lecnum]
\newtheorem{lemma}[theorem]{Lemma}

\newtheorem{proposition}[theorem]{Proposition}
\newtheorem{claim}[theorem]{Claim}
\newtheorem{corollary}[theorem]{Corollary}
\newtheorem{definition}[theorem]{Definition}
% \newenvironment{proof}{{\bf Proof:}}{\hfill\rule{2mm}{2mm}}

% **** IF YOU WANT TO DEFINE ADDITIONAL MACROS FOR YOURSELF, PUT THEM HERE:

\newcommand\E{\mathbb{E}}

\begin{document}
%FILL IN THE RIGHT INFO.
%\lecture{**LECTURE-NUMBER**}{**DATE**}{**LECTURER**}{**SCRIBE**}
\lecture{1}{信息与比特}{华园(202000120027))}{2022.2.24}
%\footnotetext{These notes are partially based on those of Nigel Mansell.}

% **** YOUR NOTES GO HERE:

% Some general latex examples and examples making use of the
% macros follow.
%**** IN GENERAL, BE BRIEF. LONG SCRIBE NOTES, NO MATTER HOW WELL WRITTEN,
%**** ARE NEVER READ BY ANYBODY.

\section*{Problem 1.} % Don't be this informal in your notes!
Proof the following atatements are true for a 0-1 Boolean algebra.
\begin{Solution}
\begin{align*}

 (1)\quad a·(a+(b·c)) &= a·a+a·(b·c) \\

&= a+a·(b·c) \\

&= a(1+b·c) \\

&=a 
\end{align*}

\begin{align*}
(2)\quad (a·b)+(\overline{a}+\overline{b}) &=(a·b)+\overline{(a·b)} \\
&=1
\end{align*}
\end{Solution}




\section*{Problem 2.}
Coin flips.A fair coin is flipped until the first head occurs. Let X denote the number of flips required. Find the 
entropy H(X) in bits.\\
\begin{Solution}
Suppose the probability distribution of random variable X is P(X).
\begin{align*}
P(X=1) &=\frac{1}{2} \\
P(X=2) &=(\frac{1}{2})^2   \\
P(X=3) &=(\frac{1}{2})^3   \\
P(X=4) &=(\frac{1}{2})^4   \\
\cdots       \\
P(X=n) &=(\frac{1}{2})^n   \\
\end{align*}
According to the definition of entropy, we can get the following results:
\end{Solution}
\begin{align*}
H(X) &=-\sum_{n=1}^\infty p(x)log_{2}p(x) \\
&=-\sum_{n=1}^\infty(\frac{1}{2})^nlog_{2}(\frac{1}{2})^n\\
&=\sum_{n=1}^\infty n(\frac{1}{2})^n\\
&=\frac{1/2}{(1-1/2)^2}\\
&=2(bits)
\end{align*}
\section*{Problem 3.}
\begin{Solution}
\begin{align*}
m(a,b)=\overline{a}+\overline{b}=\overline{ab}\quad&\quad
P(a,b)=\overline{a}b+a\overline{b}\\
m(a,a)=\overline{a}\qquad m(\overline{a},b)&=\overline{(\overline{a}b)}\qquad
m(a,\overline{b})=\overline{(a\overline{b})}\\
m(\overline{(\overline{a}b)},\overline{(a\overline{b})})&=\overline{a}b+a\overline{b}=P(a,b)
\end{align*}
\end{Solution}
% **** THIS ENDS THE EXAMPLES. DON'T DELETE THE FOLLOWING LINE:
\begin{center}
  \begin{tikzpicture}
\draw  [black](1,0)--(5,0);
\draw  [black](1,0)--(1,2);
\draw  [black](1,2)--(5,2);
\draw  [black](5,2)--(5,0);

\draw  [black](1,3)--(5,3);
\draw  [black](5,3)--(5,5);
\draw  [black](5,5)--(1,5);
\draw  [black](1,3)--(1,5);

\draw  [black](-1,0)--(-5,0);
\draw  [black](-1,0)--(-1,2);
\draw  [black](-1,2)--(-5,2);
\draw  [black](-5,2)--(-5,0);

\draw  [black](-1,3)--(-5,3);
\draw  [black](-5,3)--(-5,5);
\draw  [black](-5,5)--(-1,5);
\draw  [black](-1,3)--(-1,5);

\draw  [black](3,3)--(3,2.5);
\draw  [black](3,2.5)--(4,2.5);
\draw  [black](2,2.5)--(0.5,2.5);
\draw  [black](0.5,2.5)--(0.5,5.75);
\draw  [black](0.5,5.75)--(-2,5.75);
\draw  [black,->](2,2.5)--(2,2);
\draw  [black,->](4,2.5)--(4,2);

\draw  [black](-3,3)--(-3,2.5);
\draw  [black](-3,2.5)--(-4,2.5);
\draw  [black](-2,2.5)--(-0.5,2.5);
\draw  [black](-0.5,2.5)--(-0.5,5.5);
\draw  [black](-0.5,5.5)--(2,5.5);
\draw  [black,->](-2,2.5)--(-2,2);
\draw  [black,->](-4,2.5)--(-4,2);

\draw  [black](3,6)--(3,5.5);
\draw  [black](3,5.5)--(2,5.5);
\draw  [black](3,5.5)--(4,5.5);
\draw  [black,->](2,5.5)--(2,5);
\draw  [black,->](4,5.5)--(4,5);

\draw  [black](-3,6)--(-3,5.75);
\draw  [black](-3,5.75)--(-2,5.75);
\draw  [black](-3,5.75)--(-4,5.75);
\draw  [black,->](-2,5.75)--(-2,5);
\draw  [black,->](-4,5.75)--(-4,5);


\draw  [black](-2,-1)--(2,-1);
\draw  [black](-2,-3)--(2,-3);
\draw  [black](-2,-1)--(-2,-3);
\draw  [black](2,-1)--(2,-3);

\draw  [black](3,0)--(3,-0.5);
\draw  [black](-3,0)--(-3,-0.5);
\draw  [black](-3,-0.5)--(-1,-0.5);
\draw  [black,](3,-0.5)--(1,-0.5);
\draw  [black,->](1,-0.5)--(1,-1);
\draw  [black,->](-1,-0.5)--(-1,-1);

\draw  [black,->](0,-3)--(0,-3.5);
\put(75,25){\small\bfseries\color{blue}{$m(a,b)$}}
\put(75,110){\small\bfseries\color{blue}{$m(a,b)$}}
\put(85,175){\small\bfseries\color{blue}{$b$}}
    
\put(-95,25){\small\bfseries\color{blue}{$m(a,b)$}}
\put(-95,110){\small\bfseries\color{blue}{$m(a,b)$}}
\put(-87,175){\small\bfseries\color{blue}{$a$}}

\put(-15,-60){\small\bfseries\color{blue}{$m(a,b)$}}
\put(-15,-110){\small\bfseries\color{blue}{$P(a,b)$}}
  \end{tikzpicture}
\end{center}
\end{document}





